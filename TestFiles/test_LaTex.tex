\documentclass[12pt]{article}
\usepackage{amsmath, amssymb, amsthm, mathtools}
\usepackage{physics}
\usepackage{geometry}
\usepackage{tikz}
\usepackage{array}
\usepackage{longtable}
\usepackage{bm}
\geometry{a4paper, margin=1in}

\newtheorem{theorem}{Theorem}
\newtheorem{lemma}{Lemma}

\begin{document}

\section{Extremely Long and Complex LaTeX Example}

We study a function $f:\mathbb{R}^n \to \mathbb{R}$ that is twice differentiable with continuous derivatives.  
Let the gradient and Hessian be denoted by
\[
\nabla f(x), \qquad H(x) = \nabla^2 f(x).
\]

The Taylor expansion up to third order around $x_0$ is
\begin{align}
f(x) &= f(x_0) + \nabla f(x_0)^\top (x-x_0) 
    + \frac{1}{2}(x-x_0)^\top H(x_0)(x-x_0) \nonumber \\
    &\quad + \frac{1}{6} \sum_{i,j,k=1}^n 
        \frac{\partial^3 f}{\partial x_i \partial x_j \partial x_k}(x_0)
        (x_i-x_{0,i})(x_j-x_{0,j})(x_k-x_{0,k}) + O(\|x-x_0\|^4).
\end{align}

\subsection{A Very Long Theorem}

\begin{theorem}[Generalized Spectral Decomposition]
Let $A$ be a real symmetric matrix in $\mathbb{R}^{n\times n}$.  
Then there exists an orthogonal matrix $Q$ and a diagonal matrix $\Lambda$ such that
\[
A = Q \Lambda Q^\top.
\]
Moreover, if $A$ is positive definite, then for any vector $x$,
\[
x^\top A x = \sum_{i=1}^n \lambda_i (q_i^\top x)^2,
\]
where $\lambda_i$ are eigenvalues and $q_i$ eigenvectors.
\end{theorem}

\begin{proof}
(The classical long proof omitted.) For illustration, note that real symmetric matrices are diagonalizable by orthonormal eigenvectors.  
Thus $A q_i = \lambda_i q_i$ for each $i$, and arranging these as columns of $Q$ yields $A = Q \Lambda Q^\top$.
\end{proof}

\subsection{Long Derivation: Moore--Penrose Pseudoinverse}

For a non-square matrix $A\in\mathbb{R}^{m\times n}$, the pseudoinverse is
\[
A^+ = (A^\top A)^{-1} A^\top \quad \text{if $A$ has full column rank}.
\]

More generally, with SVD:
\[
A = U\Sigma V^\top, \qquad 
A^+ = V \Sigma^+ U^\top.
\]

\subsection{A Massive Matrix Example}

\[
M = 
\begin{pmatrix}
1 & 2 & 3 & 4 & 5 \\
0 & 1 & -1 & 2 & -3 \\
4 & 0 & 2 & -1 & 7 \\
1 & -2 & 3 & 0 & 0 \\
-3 & 5 & 1 & 1 & 2
\end{pmatrix}.
\]

\subsection{A Giant Multiline Equation Spanning Pages}

\begin{align}
F(t) &= \int_{0}^{t} 
\left[
    x^3 e^{-x^2}
    + \frac{\sin(x)}{1+x^2}
    + \ln(1+x)
    + \sum_{k=1}^{\infty}
      \frac{(-1)^k x^{2k}}{k!}
\right]\dd{x} \nonumber \\
&\quad 
+ \lim_{n\to\infty} 
\left(
    \sum_{i=1}^{n}
    \frac{(-1)^i}{i}
    \int_0^1 u^{i-1}(1-u)^{i-1}\dd{u}
\right)
+ \int_{-\infty}^{\infty} e^{-x^4/2}\dd{x}. 
\end{align}

\subsection{A Three-Dimensional Integral}

\[
I = \iiint_{x^2 + y^2 + z^2 \le 1} 
\frac{x^2 + y^2 + z^2}{1 + xyz}
\dd{x}\dd{y}\dd{z}.
\]

\subsection{Piecewise and Limit Mixture}

\[
h(x)=
\begin{cases}
x^2\sin(1/x) + x^3, & x\neq 0, \\
\displaystyle \lim_{t\to 0} t^2\sin(1/t), & x=0.
\end{cases}
\]

\subsection{An Inner Product Space Example}

For $f,g\in L^2[0,1]$, define
\[
\langle f,g\rangle = \int_0^1 f(x)g(x)\dd{x}.
\]

Then the Fourier series expansion of a periodic function $f$ is
\[
f(x)\sim \frac{a_0}{2} + \sum_{n=1}^{\infty}(a_n\cos nx + b_n\sin nx).
\]

\subsection{TikZ Example}

\begin{center}
\begin{tikzpicture}[scale=1.1]
\draw[->] (-0.5,0) -- (3.5,0) node[right]{$x$};
\draw[->] (0,-0.5) -- (0,2.5) node[above]{$y$};
\draw[thick, blue] (0,0) .. controls (1,2) .. (3,1);
\node at (1.5,2) {$y=f(x)$};
\end{tikzpicture}
\end{center}

\subsection{A Long Table for Layout Stress Testing}

\begin{longtable}{|c|c|c|c|}
\hline
$n$ & $a_n$ & $b_n$ & Remark \\
\hline
1 & $\frac{1}{2}$ & $0$ & base term\\
2 & $\frac{1}{4}$ & $-\frac{1}{3}$ & oscillatory\\
3 & $0$ & $\frac{1}{5}$ & slow decay\\
4 & $-\frac{1}{8}$ & $0$ & symmetry\\
5 & $\frac{1}{16}$ & $-\frac{1}{7}$ & etc.\\
\hline
\end{longtable}

\subsection{Another Long Multi-step Derivation}

\begin{align}
R(x) 
&= \sum_{n=1}^\infty \frac{(-1)^{n-1} x^n}{n}
    + \int_0^x t^2 e^{-t} \dd{t}
    + \int_{0}^{\infty} \frac{\cos(tx)}{1+t^4}\dd{t} \\
&= \ln(1+x)
    - (x^2+2x+2)e^{-x}
    + \frac{\pi}{\sqrt{2}} e^{-x/\sqrt{2}}
      \left[
        \sin\left(\frac{x}{\sqrt{2}}\right)
        + 
        \cos\left(\frac{x}{\sqrt{2}}\right)
      \right].
\end{align}

\end{document}
